\documentclass[10pt]{article}
\usepackage[utf8]{inputenc}
\usepackage[activeacute,spanish]{babel}
\usepackage[left=1.5cm,top=1.5cm,right=1.5cm, bottom=1.5cm,letterpaper, includeheadfoot]{geometry}

\usepackage{amssymb, amsmath, amsthm}
\usepackage{graphicx}
\usepackage{hyperref}
\usepackage{lmodern,url}
\usepackage{paralist} %util para listas compactas
\usepackage{xcolor}
\usepackage{bbm}
\usepackage{mathrsfs}
\usepackage{bbm}

%========PAQUETES AGREGADOS===========
%Pseudocodigo
\usepackage{pseudocode}
\usepackage[portuguese, boxruled]{algorithm2e}
\usepackage{wrapfig}
\usepackage{multicol}
\usepackage{graphicx}
\usepackage{caption}
\usepackage{subcaption}
%\captionsetup[table]{labelformat=empty}
\captionsetup[subfigure]{labelformat=empty}
\usepackage{cancel}
\usepackage{tikz}
\def\checkmark{\tikz\fill[scale=0.4](0,.35) -- (.25,0) -- (1,.7) -- (.25,.15) -- cycle;} 
%====================================

\usepackage{fancyhdr}
\pagestyle{fancy}
\fancypagestyle{plain}{%
\fancyhf{}
\lhead{\footnotesize\itshape\bfseries\rightmark}
\rhead{\footnotesize\itshape\bfseries\leftmark}
}


% macros
\newcommand{\Q}{\mathbb Q}
\newcommand{\R}{\mathbb R}
\newcommand{\N}{\mathbb N}
\newcommand{\Z}{\mathbb Z}
\newcommand{\C}{\mathbb C}
\newcommand{\BigO}{\mathcal{O}}
%Teoremas, Lemas, etc.
\theoremstyle{plain}
\newtheorem{teo}{Teorema}
\newtheorem{lem}{Lema}
\newtheorem{prop}{Proposición}
\newtheorem{cor}{Corolario}
\newtheorem{obs}{Observación}
\newtheorem{ej}{Ejemplo}
\renewcommand{\qedsymbol}{\rule{0.7em}{0.7em}}
\renewenvironment{proof}{{\bfseries \noindent Demostración}}{ \qed \\}


\theoremstyle{definition}
\newtheorem{defi}{Definición}
% fin macros


\newcommand{\catnum}{18} %numero de catedra
\newcommand{\fecha}{8 de Noviembre 2016 }

%%%%%%%%%%%%%%%%%%

%Macros para este documento
\newcommand{\cin}{\operatorname{cint}}



\begin{document}
%Encabezado
\fancyhead[L]{Facultad de Ciencias Físicas y Matemáticas}
\fancyhead[R]{Universidad de Chile}
\vspace*{-1.2 cm}
\begin{minipage}{0.6\textwidth}
\begin{flushleft}
\hspace*{-0.5cm}\textbf{MA3402-1 Estadística. Primavera 2016}\\
\hspace*{-0.5cm}\textbf{Profesor:} Raul Gouet\\
\hspace*{-0.5cm}\textbf{Escriba:} Manuel Cáceres\\
\hspace*{-0.5cm}\textbf{Fecha:} \fecha
\end{flushleft}
\end{minipage}
\begin{minipage}{0.36\textwidth}
\begin{flushright}
\includegraphics[scale=0.3]{imagenes/fcfm_dcc}
\end{flushright}
\end{minipage}
\bigskip
%Fin encabezado

\begin{center}
\LARGE\textbf{Clase \catnum}
\end{center}
Veremos un ejemplo de Teoría de decisión aplicado a test de hipótesis.\\

Recordemos los elementos: $\underbrace{f(X|\theta)}_{sale\ X} , \underbrace{\pi(\theta)}_{priori}, \underbrace{L(\cdot,\cdot)}_{f\ de\ perdida}, \underbrace{\delta(\cdot)}_{regla\ de\ decision}$
\begin{align*}
\delta\colon \mathfrak{X} \mapsto D, L(\cdot,\cdot)\colon D \times \Theta \mapsto \mathbb{R}
\end{align*}

Riesgo
\begin{align*}
R_{\delta}(\theta) = \mathbb{E}_{\theta}(L(\delta(X),\theta)) = \int_{\mathfrak{X}} L(\delta(X), \theta) f(X|\theta) dx
\end{align*}

$\delta$ domina a $\delta'$ si $R_{\delta}(\theta) \le R_{\delta'}(\theta) (\exists \theta con <)$\\

$\delta$ es admisible si $\not \exists \delta'$ que la domina.\\

Regla minimax (muy conservadora), criterio $\mapsto \max_{\theta}R_{\delta}(\theta)$.\\

$\delta$ es mejor que $\delta'$ si $\max_{\theta}R_{\delta}(\theta) \le \max_{\theta}R_{\delta'}(\theta)$.\\

$\delta$ es minimax si $\max_{\theta}R_{\delta}(\theta) \le \max_{\theta}R_{\delta'}(\theta) \forall \delta' \ldots$

\section{Regla de Bayes}
 Riesgo medio : $\bar{R_{\delta}} = \int_{\Theta}R_{\delta}(\theta)\pi(\theta)d\theta$.\\
 
 Regla de Bayes $\delta$: $\bar{R_{\delta}} \le \bar{R_{\delta'}}, \forall \delta' \in AlgunaColeccion$.\\
 
 Un ejemplo de esto es el EIVUM.\\
 
 El problema $\min_{\delta}\bar{R_{\delta}}$ es un problema de cálculo variacional(dificil).\\
 
 Se puede reducir a $\infty$ problemas en $\mathbb{R}$, via Fubini.\\
 
 \begin{align*}
 \bar{R_{\delta}} &= \int_{\Theta}R_{\delta}(\theta)\pi(\theta)d\theta 0 \int_{\Theta}\int_{\mathfrak{X}} L(\delta(X),\theta) f(X|\theta)dx \pi(\theta)d\theta\\
 &= \int_{\mathfrak{X}}\left(\int_{\theta}\right)
 \end{align*}
\end{document}