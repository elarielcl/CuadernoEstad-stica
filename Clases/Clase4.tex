\documentclass[10pt]{article}
\usepackage[utf8]{inputenc}
\usepackage[activeacute,spanish]{babel}
\usepackage[left=1.5cm,top=1.5cm,right=1.5cm, bottom=1.5cm,letterpaper, includeheadfoot]{geometry}

\usepackage{amssymb, amsmath, amsthm}
\usepackage{graphicx}
\usepackage{hyperref}
\usepackage{lmodern,url}
\usepackage{paralist} %util para listas compactas
\usepackage{xcolor}
\usepackage{bbm}


%========PAQUETES AGREGADOS===========
%Pseudocodigo
\usepackage{pseudocode}
\usepackage[portuguese, boxruled]{algorithm2e}
\usepackage{wrapfig}
\usepackage{multicol}
\usepackage{graphicx}
\usepackage{caption}
\usepackage{subcaption}
%\captionsetup[table]{labelformat=empty}
\captionsetup[subfigure]{labelformat=empty}
\usepackage{cancel}
%====================================

\usepackage{fancyhdr}
\pagestyle{fancy}
\fancypagestyle{plain}{%
\fancyhf{}
\lhead{\footnotesize\itshape\bfseries\rightmark}
\rhead{\footnotesize\itshape\bfseries\leftmark}
}


% macros
\newcommand{\Q}{\mathbb Q}
\newcommand{\R}{\mathbb R}
\newcommand{\N}{\mathbb N}
\newcommand{\Z}{\mathbb Z}
\newcommand{\C}{\mathbb C}
\newcommand{\BigO}{\mathcal{O}}
%Teoremas, Lemas, etc.
\theoremstyle{plain}
\newtheorem{teo}{Teorema}
\newtheorem{lem}{Lema}
\newtheorem{prop}{Proposición}
\newtheorem{cor}{Corolario}
\newtheorem{obs}{Observación}
\newtheorem{ej}{Ejemplo}
\renewcommand{\qedsymbol}{\rule{0.7em}{0.7em}}
\renewenvironment{proof}{{\bfseries \noindent Demostración}}{ \qed \\}


\theoremstyle{definition}
\newtheorem{defi}{Definición}
% fin macros


\newcommand{\catnum}{4} %numero de catedra
\newcommand{\fecha}{13 de Septiembre 2016 }

%%%%%%%%%%%%%%%%%%

%Macros para este documento
\newcommand{\cin}{\operatorname{cint}}



\begin{document}
%Encabezado
\fancyhead[L]{Facultad de Ciencias Físicas y Matemáticas}
\fancyhead[R]{Universidad de Chile}
\vspace*{-1.2 cm}
\begin{minipage}{0.6\textwidth}
\begin{flushleft}
\hspace*{-0.5cm}\textbf{MA3402-1 Estadística. Primavera 2016}\\
\hspace*{-0.5cm}\textbf{Profesor:} Raul Gouet\\
\hspace*{-0.5cm}\textbf{Escriba:} Manuel Cáceres\\
\hspace*{-0.5cm}\textbf{Fecha:} \fecha
\end{flushleft}
\end{minipage}
\begin{minipage}{0.36\textwidth}
\begin{flushright}
\includegraphics[scale=0.3]{imagenes/fcfm_dcc}
\end{flushright}
\end{minipage}
\bigskip
%Fin encabezado

\begin{center}
\LARGE\textbf{Clase \catnum}
\end{center}

\section{Suficiencia y EIVUM}
Tenemos el concepto de estadístico suficiente $T$ ($T\colon \mathfrak{X} \mapsto \mathcal{T}$) para una familia $\mathcal{P}=\{\mathbb{P}_{\theta}\colon \theta \in \Theta\}$.\\
También $T(X)$ se llama estadístico.\\
La ley de $X$ condicionada en $T(X)$ no depende de $\theta$, $\Leftrightarrow \mathbb{P}_{\theta}(X \in A|T(X)) = H(A,T(X))$.\\
Famoso teorema de factorización: $f_{\theta}(X) = g_{\theta}((T(x)))h(x)$.\\
\begin{align*}
\mathbb{\theta}(X\in A) &= \int_{A}f_{\theta}(x)dx, \forall A \in \mathcal{B}(\mathfrak{X})\\
&= \sum_{X\in A}f_{\theta}(X)
\end{align*}

\begin{ej} La familia exponencial.\\
\begin{align*}
f_{\theta}(x) &= c(\theta) e^{\sum \theta_{i}t_{i}(x)}h(x)\\
T(X) &= \sum (t_{1}(x), \ldots, t_{k}(x)) , \text{es un suficiente}
\end{align*}
\end{ej}

También definimos la idea de suficiencia minimal que significa la mayor reducción posible de información.\\
Si tenemos 2 estadísticos suficientes $S,T$ tales que $S=f(T)$ entonces preferimos $S$.\\
Si $S$ es función de cualquier otro estadístico suficiente, entonces $S$ es suficiente minimal (por definición).\\
¿Existe?, Sí, si existe densidad $f_{\theta}$(continuo o discreto).\\
Si $\frac{f_{\theta}(x)}{f_{\theta}(y)}$ no depende de $\theta$ decimos que $x\mathcal{R}y$. Esta relación es de equivalencia y forma clases de equivalencia y $S$ se define constante en estas clases de equivalencia.

\section{Teorema de Rao-Blackwell}
Todo estimador insesgado, por malo que sea(con varianza finita) se puede mejorar mediante una ``Rao-Blackwelización'', algo así como una proyección sobre un $T$ suficiente.\\
En concreto, si $\tilde{g}$ es insesgado para $g(\theta)$, calculamos $\mathbb{E}_{\theta}(\tilde{g}(X)|T(X))$ que no depende de $\theta$ y que es una variable aleatoria que solo depende de $T(X)$, entonces existe una función $\hat{g}$ tal que:
\begin{align*}
\hat{g}(T(X)) = \mathbb{E}_{\theta}(\tilde{g}(X)|T(X))
\end{align*}
Por una propiedad de la esperanza condicional (la esperanza de la esperanza condicional, es la eperanza $\mathbb{E}(\mathbb{E}(X|Y)) = \mathbb{E}(X)$). \\
Resulta que:
\begin{align*}
\mathbb{E}_{\theta}(\tilde{g}(T(X))) &= \mathbb{E}_{\theta}(\mathbb{E}_{\theta}(\hat{g}(X)|T(X)))\\
&= \mathbb{E}_{\theta}(\hat{g}(X))\\
&= g(\theta)
\end{align*}
Además TRB dice que:
\begin{align*}
\mathbb{V}_{\theta}(\hat{g}(T(X))) \le \mathbb{V}_{\theta}(\hat{g}(X)),\ \forall \theta \in \Theta
\end{align*}
Notar que la Re-Rao-Blackwelización:
\begin{align*}
\mathbb{E}_{\theta}(\hat{g}(T(X)|T(X)) = \hat{g}(T(X)), :(
\end{align*}
Quizás existe otro suficiente y calculamos :
\begin{align*}
\mathbb{E}_{\theta}(\hat{g}(T(X))|S(X))
\end{align*}
\section{Completitud}
\end{document}