\documentclass[10pt]{article}
\usepackage[utf8]{inputenc}
\usepackage[activeacute,spanish]{babel}
\usepackage[left=1.5cm,top=1.5cm,right=1.5cm, bottom=1.5cm,letterpaper, includeheadfoot]{geometry}

\usepackage{amssymb, amsmath, amsthm}
\usepackage{graphicx}
\usepackage{hyperref}
\usepackage{lmodern,url}
\usepackage{paralist} %util para listas compactas
\usepackage{xcolor}

%========PAQUETES AGREGADOS===========
%Pseudocodigo
\usepackage{pseudocode}
\usepackage[portuguese, boxruled]{algorithm2e}
\usepackage{wrapfig}
\usepackage{multicol}
\usepackage{graphicx}
\usepackage{caption}
\usepackage{subcaption}
%\captionsetup[table]{labelformat=empty}
\captionsetup[subfigure]{labelformat=empty}
\usepackage{cancel}
%====================================

\usepackage{fancyhdr}
\pagestyle{fancy}
\fancypagestyle{plain}{%
\fancyhf{}
\lhead{\footnotesize\itshape\bfseries\rightmark}
\rhead{\footnotesize\itshape\bfseries\leftmark}
}


% macros
\newcommand{\Q}{\mathbb Q}
\newcommand{\R}{\mathbb R}
\newcommand{\N}{\mathbb N}
\newcommand{\Z}{\mathbb Z}
\newcommand{\C}{\mathbb C}
\newcommand{\BigO}{\mathcal{O}}
%Teoremas, Lemas, etc.
\theoremstyle{plain}
\newtheorem{teo}{Teorema}
\newtheorem{lem}{Lema}
\newtheorem{prop}{Proposición}
\newtheorem{cor}{Corolario}
\newtheorem{obs}{Observación}
\newtheorem{ej}{Ejemplo}
\renewcommand{\qedsymbol}{\rule{0.7em}{0.7em}}
\renewenvironment{proof}{{\bfseries \noindent Demostración}}{ \qed \\}


\theoremstyle{definition}
\newtheorem{defi}{Definición}
% fin macros


\newcommand{\catnum}{2} %numero de catedra
\newcommand{\fecha}{6 de Septiembre 2016 }

%%%%%%%%%%%%%%%%%%

%Macros para este documento
\newcommand{\cin}{\operatorname{cint}}



\begin{document}
%Encabezado
\fancyhead[L]{Facultad de Ciencias Físicas y Matemáticas}
\fancyhead[R]{Universidad de Chile}
\vspace*{-1.2 cm}
\begin{minipage}{0.6\textwidth}
\begin{flushleft}
\hspace*{-0.5cm}\textbf{MA3402-1 Estadística. Primavera 2016}\\
\hspace*{-0.5cm}\textbf{Profesor:} Raul Gouet\\
\hspace*{-0.5cm}\textbf{Escriba:} Manuel Cáceres\\
\hspace*{-0.5cm}\textbf{Fecha:} \fecha
\end{flushleft}
\end{minipage}
\begin{minipage}{0.36\textwidth}
\begin{flushright}
\includegraphics[scale=0.3]{imagenes/fcfm_dcc}
\end{flushright}
\end{minipage}
\bigskip
%Fin encabezado

\begin{center}
\LARGE\textbf{Clase \catnum}
\end{center}

Tenemos un modelo que consiste en una familia $\mathcal{P}$ de medida de probabilidad sobre $(\Omega, \mathcal{F})$.\\

Escribimos:
\begin{itemize}
\item $\mathcal{P}=\{\mathbb{P}_{\theta}\colon \theta \in \Theta \}$, en donde a $\theta$ le llamamos parámetro del modelo y $\Theta$ espacio de parámetros.
\end{itemize}
Cuando $\Theta \subseteq \mathbb{R}^k$, tenemos $\theta = (\theta_{1}, \theta_{2}, \ldots, \theta_{k})$ y los $\theta_{i}$ son parámetros reales. En este caso diremos que nuestro modelo es paramétrico (pues hay un número finito de parámetros).\\

Para nosotros $\theta$ es un parámetro fijo desconocido.\\
Podemos observar $X$ (que es un objeto aleatorio) a valores en $\mathfrak{X}$ (espacio medible, típicamente $\mathbb{R}^n$).
\begin{align*}
X \colon (\Omega, \mathcal{F}, ?) \mapsto (\mathfrak{X}, \mathcal{B}(\mathfrak{X}), inducida)
\end{align*}

Supondremmos que está actuando una probabilidad $\mathbb{P}_{\theta} \in \mathcal{P}$, pero no sabemos cual. Queremos adivinar cual es en base a la observación de $X$.\\

Si hemos visto $X=x$, entonces postularemos que la probabilidad en acción es tal.\\

El problema de estimación consiste en usar la información entregada por $X$ para calcular aproximadamente(estimar) una función $g(\theta)$ del parámetro. Esto se consigue definiendo los estadísticos y estimadores.

\section{Estadístico}
Se define como estadístico a cualquier función medible $T: \mathfrak{X} \mapsto \mathcal{T}$(otro espacio medible).
El estadístico $T$ puede servir para ``resumir'' la información.\\

Por ejemplo, si $\mathfrak{X} = \mathbb{R}^n,\ x = (x_{1}, x_{2}, \ldots, x_{n})$ entonces son estadísticos:
\begin{itemize}
\item $T(x) = \sum_{i=1}^n{x_{i}}$
\item $S(x) = max\{x_{1}, x_{2}, \ldots, x_{n}\}$
\item $H(x) = \left(\sum_{i=1}^n{x_{i}}, \sum_{i=1}^n{x_{i}^2}\right)$
\end{itemize}
Estos estadísticos son muy comunes.\\

Por extensión diremos que la variable aleatoria $T(X)$ (composición) es un estadístico.

\section{Estimador}
Un estimador de $g(\theta)$ es un estadístico (habitualmente denotado $\hat{g}$ o $\tilde{g}$)	cuyos valores se parezcan a $g(\theta)$. También diremos que $\hat{g}(X)$ es estimador. Pero a los valores $\hat{g}(x)$ les llamaremos estimaciones.

\begin{ej} Modelo de Bernoulli con $\theta \in \left[0,1\right],\ X = (X_{1}, X_{2}, \ldots, X_{n}),\ \mathfrak{X} = \{0,1\}^n$.\\
Donde $\mathbb{P}_{\theta}\left[X=x\right] = \theta^{\sum{x_{i}}}(1-\theta)^{n-\sum{x_{i}}}$.\\
Sea $T(x) = \sum_{i=1}^n{x_{i}}$ un estadístico, y $g(\theta) = \theta$ la función a estimar.\\

Sean $\hat{g_{1}}(x) = \frac{1}{n}\sum_{i=1}^n{x_{i}} = \frac{1}{n}T(x)$ y $\hat{g_{2}}(x) = x_{1}$ dos estimadores de $g(\theta)$.\\
Si observamos $x=(1,1,0,0,1,0)$ entonces $\hat{g_{1}}((1,1,0,0,1,0)) = \frac{1}{2}$ y $\hat{g_{2}}((1,1,0,0,1,0)) = 1$ son dos estimaciones de $\theta (g(\theta))$.\\
¿Cómo medimos la calidad de un estimador?

\section{Calidad de un estimador}

\end{ej}
\end{document}