\documentclass[10pt]{article}
\usepackage[utf8]{inputenc}
\usepackage[activeacute,spanish]{babel}
\usepackage[left=1.5cm,top=1.5cm,right=1.5cm, bottom=1.5cm,letterpaper, includeheadfoot]{geometry}

\usepackage{amssymb, amsmath, amsthm}
\usepackage{graphicx}
\usepackage{hyperref}
\usepackage{lmodern,url}
\usepackage{paralist} %util para listas compactas
\usepackage{xcolor}
\usepackage{bbm}
\usepackage{mathrsfs}
\usepackage{bbm}

%========PAQUETES AGREGADOS===========
%Pseudocodigo
\usepackage{pseudocode}
\usepackage[portuguese, boxruled]{algorithm2e}
\usepackage{wrapfig}
\usepackage{multicol}
\usepackage{graphicx}
\usepackage{caption}
\usepackage{subcaption}
%\captionsetup[table]{labelformat=empty}
\captionsetup[subfigure]{labelformat=empty}
\usepackage{cancel}
\usepackage{tikz}
\def\checkmark{\tikz\fill[scale=0.4](0,.35) -- (.25,0) -- (1,.7) -- (.25,.15) -- cycle;} 
%====================================

\usepackage{fancyhdr}
\pagestyle{fancy}
\fancypagestyle{plain}{%
\fancyhf{}
\lhead{\footnotesize\itshape\bfseries\rightmark}
\rhead{\footnotesize\itshape\bfseries\leftmark}
}


% macros
\newcommand{\Q}{\mathbb Q}
\newcommand{\R}{\mathbb R}
\newcommand{\N}{\mathbb N}
\newcommand{\Z}{\mathbb Z}
\newcommand{\C}{\mathbb C}
\newcommand{\BigO}{\mathcal{O}}
%Teoremas, Lemas, etc.
\theoremstyle{plain}
\newtheorem{teo}{Teorema}
\newtheorem{lem}{Lema}
\newtheorem{prop}{Proposición}
\newtheorem{cor}{Corolario}
\newtheorem{obs}{Observación}
\newtheorem{ej}{Ejemplo}
\renewcommand{\qedsymbol}{\rule{0.7em}{0.7em}}
\renewenvironment{proof}{{\bfseries \noindent Demostración}}{ \qed \\}


\theoremstyle{definition}
\newtheorem{defi}{Definición}
% fin macros


\newcommand{\catnum}{11} %numero de catedra
\newcommand{\fecha}{13 de Septiembre 2016 }

%%%%%%%%%%%%%%%%%%

%Macros para este documento
\newcommand{\cin}{\operatorname{cint}}



\begin{document}
%Encabezado
\fancyhead[L]{Facultad de Ciencias Físicas y Matemáticas}
\fancyhead[R]{Universidad de Chile}
\vspace*{-1.2 cm}
\begin{minipage}{0.6\textwidth}
\begin{flushleft}
\hspace*{-0.5cm}\textbf{MA3402-1 Estadística. Primavera 2016}\\
\hspace*{-0.5cm}\textbf{Profesor:} Raul Gouet\\
\hspace*{-0.5cm}\textbf{Escriba:} Manuel Cáceres\\
\hspace*{-0.5cm}\textbf{Fecha:} \fecha
\end{flushleft}
\end{minipage}
\begin{minipage}{0.36\textwidth}
\begin{flushright}
\includegraphics[scale=0.3]{imagenes/fcfm_dcc}
\end{flushright}
\end{minipage}
\bigskip
%Fin encabezado

\begin{center}
\LARGE\textbf{Clase \catnum}
\end{center}
\section{Hipótesis Compuestas}
Recordemos quel TNP es el óptimo para el problema del tipo $H_{0}: \theta = \theta_{0}$ vs $H_{1}: \theta = \theta_{1}$ con $\Theta = \{\theta_{0}, \theta_{1}\}$ en el sentido de tener la mejor potencia posible en la clase $\tau_{\alpha}$ (de nivel $\alpha$).\\
	
\textbf{¿Cómo abordar los problemas más realistas (cercanos a la aplicaciones) donde $H_{0}$ o $H_{1}$ o ambos son compuestas?}\\

Veremos que el TNP puede servir (sirve) en problemas compuestos unilaterales.\\
Recordemos que el TNP $\phi^*$ para $H_{0}: \mu = \mu_{0}$ vs $H_{1}: \mu = \mu_{1}, \mu_{1}>\mu_{0}$ en el modelo gaussiano, basados en la misma muestra iid $X_{1},\ldots,X_{n}$.\\
Sabemos que $\phi^*(X)=1 \Leftrightarrow \bar{X} \ge \mu_{0} + \frac{\sigma}{\sqrt{n}}\phi^{-1}(1-\alpha)$.\\
Sabemos que $\phi^* \in \tau_{\alpha} (\alpha_{\phi^*}(\mu_{0}) = \alpha)$ y que:
\begin{align*}
\alpha_{\phi^*}(\mu_{1}) = \underbrace{(1-\phi)}_{funcion} \left(\frac{\sqrt{n}(\mu_{0}-\mu_{1})}{\sigma}+ \phi^{-1}(1-\alpha)\right)
\end{align*}
con $\sigma_{\phi^*}(\mu_{1}) \ge \alpha_{\phi}(\mu_{1}), \forall \phi \in \tau_{\alpha}$.\\

Pensemos en el problema unilateral (una cola) $P'$, $H_{0}: \underbrace{\mu = \mu_{0}}_{simple}$ vs $H_{1}: \underbrace{\mu > \mu_{0}}_{compuesta}$, se entiende que $\Theta = [\mu_{0},\infty[$.\\
Veamos que pasa si usamos el TNP $\phi^*$ en este problema. Primero notemos que $\phi^* \in \tau_{\alpha}$ (los tests de nivel $\alpha$ para $P'$).\\
\begin{align*}
 \phi \in \tau_{\alpha} & \Leftrightarrow \mathbb{P}_{\mu}(X\in \mathbb{R}_{\phi}) \le \alpha \forall \mu \in \Theta_{0} = \{\mu_{0}\}\\
 & \Leftrightarrow \mathbb{P}_{\mu_{0}}(X\in \mathbb{R}_{\phi}) \le \alpha
\end{align*}
Por definición de TNP sabemos que $\alpha_{\phi^*}(\mu) \ge \alpha_{\phi}(\mu), \forall \mu \ge \mu_{0}, \forall \phi \in \tau_{\alpha}$, de donde se ve que $\phi^*$ es test UMP de nivel $\alpha$ para el problema $P'$. SALIÓ GRATIS!\\

Se ha resuelto el problema de encontrar test UMP para $P'$ $H_{0}: \mu=\mu_{0}$ vs $H_{1}: \mu>\mu_{0}$.\\
Pasemos al problema $P''$ $H_{0}: \mu \le \mu_{0}$ vs $H_{1}: \mu>\mu_{0}$ y veamos si $\phi^*$ sigue valiendo en este caso.\\
La clase $\tau_{\alpha}$ tiene los $\phi$ tales que:
\end{document}