\documentclass[10pt]{article}
\usepackage[utf8]{inputenc}
\usepackage[activeacute,spanish]{babel}
\usepackage[left=1.5cm,top=1.5cm,right=1.5cm, bottom=1.5cm,letterpaper, includeheadfoot]{geometry}

\usepackage{amssymb, amsmath, amsthm}
\usepackage{graphicx}
\usepackage{hyperref}
\usepackage{lmodern,url}
\usepackage{paralist} %util para listas compactas
\usepackage{xcolor}

%========PAQUETES AGREGADOS===========
%Pseudocodigo
\usepackage{pseudocode}
\usepackage[portuguese, boxruled]{algorithm2e}
\usepackage{wrapfig}
\usepackage{multicol}
\usepackage{graphicx}
\usepackage{caption}
\usepackage{subcaption}
%\captionsetup[table]{labelformat=empty}
\captionsetup[subfigure]{labelformat=empty}
\usepackage{cancel}
%====================================

\usepackage{fancyhdr}
\pagestyle{fancy}
\fancypagestyle{plain}{%
\fancyhf{}
\lhead{\footnotesize\itshape\bfseries\rightmark}
\rhead{\footnotesize\itshape\bfseries\leftmark}
}


% macros
\newcommand{\Q}{\mathbb Q}
\newcommand{\R}{\mathbb R}
\newcommand{\N}{\mathbb N}
\newcommand{\Z}{\mathbb Z}
\newcommand{\C}{\mathbb C}
\newcommand{\BigO}{\mathcal{O}}
%Teoremas, Lemas, etc.
\theoremstyle{plain}
\newtheorem{teo}{Teorema}
\newtheorem{lem}{Lema}
\newtheorem{prop}{Proposición}
\newtheorem{cor}{Corolario}
\newtheorem{obs}{Observación}
\newtheorem{ej}{Ejemplo}
\renewcommand{\qedsymbol}{\rule{0.7em}{0.7em}}
\renewenvironment{proof}{{\bfseries \noindent Demostración}}{ \qed \\}


\theoremstyle{definition}
\newtheorem{defi}{Definición}
% fin macros


\newcommand{\catnum}{1} %numero de catedra
\newcommand{\fecha}{2 de Septiembre 2016 }

%%%%%%%%%%%%%%%%%%

%Macros para este documento
\newcommand{\cin}{\operatorname{cint}}



\begin{document}
%Encabezado
\fancyhead[L]{Facultad de Ciencias Físicas y Matemáticas}
\fancyhead[R]{Universidad de Chile}
\vspace*{-1.2 cm}
\begin{minipage}{0.6\textwidth}
\begin{flushleft}
\hspace*{-0.5cm}\textbf{MA3402-1 Estadística. Primavera 2016}\\
\hspace*{-0.5cm}\textbf{Profesor:} Raul Gouet\\
\hspace*{-0.5cm}\textbf{Escriba:} Manuel Cáceres\\
\hspace*{-0.5cm}\textbf{Fecha:} \fecha
\end{flushleft}
\end{minipage}
\begin{minipage}{0.36\textwidth}
\begin{flushright}
\includegraphics[scale=0.3]{imagenes/fcfm_dcc}
\end{flushright}
\end{minipage}
\bigskip
%Fin encabezado

\begin{center}
\LARGE\textbf{Clase \catnum}
\end{center}
\section{Introducción}
Este es un curso de \textbf{inferencia estadística} que es parte de la estadística matemática y de la teoría de decisiones.\\
Por otro lado, tenemos la estadística descriptiva que se encarga de describir los datos haciendo resúmenes de la información.\\
Algunos ejemplos de uso de la estadística son :
\begin{itemize}
\item IDIEM analiza los datos obtenidos de sus experimentos
\item En agricultura se utiliza sobre datos analizables
\item En los datos generados por el transporte público
\item En los datos obtenidos del comercio y la economía
\end{itemize}
A lo largo del curso veremos que en estadística \textbf{NO} existen las verdades categóricas, pues toda la teoría se basa en probabilidades.

\section{Primeros elementos y Notación}
Sea $\mathfrak{X}$ un espacio (conjunto) medible, típicamente $\mathbb{R}^n$ o un espacio métrico, donde viven los datos.\\
Denotamos por $x$ un elemento típico de $\mathfrak{X}$, y decimos que es un dato.\\
Vamos a denotar por $X$ un elemento aleatorio de $\mathfrak{X}$ (una variable o vector aleatorio en el caso de $\mathfrak{X} = \mathbb{R}\ o\ \mathbb{R}^n$). Y diremos que $x$ es una realización de $X$.\\
Lo anterior significa que $X$ es una función medible :
\begin{align*}
X \colon (\Omega, \mathcal{F}) \mapsto (\mathfrak{X}, \mathcal{B}(\mathfrak{X}))
\end{align*}

Vamos a suponer que $X$ tiene una ley de probabilidad (i.e. tiene su distribución (por ejemplo, $X$ es a valores en $\mathbb{R}$)).\\
Esta ley será una probabilidad  sobre el espacio $(\mathfrak{X}, \mathcal{B}(\mathfrak{X}))$ inducida por $X$ de una probabilidad en $(\Omega, \mathcal{F})$.\\

Si $\mathbb{P}$ es probabilidad  en $(\Omega, \mathcal{F})$, entonces la inducida por $X$ en $(\mathfrak{X}, \mathcal{B}(\mathfrak{X}))$ es $\mathbb{P} \circ X^{-1}$. Es importante notar en este punto que en estadística tenemos un modelo conocido, pero hay un parámetro desconocido y lo que estaremos haciendo es dado el $x$, encontrar la ley de probabilidad de $X$.\\

Supondremos que existe una familia de probabilidades sobre el espacio $(\Omega, \mathcal{F})$, que designamos por $\mathcal{P}$. $\mathcal{P}$ es una colección de probabilidades sobre  $(\Omega, \mathcal{F})$.\\

Supondremos también que solo una de ellas está actuando (el verdadero estado de la naturaleza) y esa probabilidad desconocida es la que refleja $X$ en su comportamiento.\\

\begin{align*}
(\Omega, \mathcal{F}, \left[\mathbb{P}\right]) \xrightarrow{X} (\mathfrak{X}, \mathcal{B}(\mathfrak{X}), \left[\mathbb{P} \circ X^{-1}\right])
\end{align*}
Nuestra misión es escoger la mejor $\mathbb{P} \in \mathcal{P}$.\\

La familia $\mathcal{P}$ se puede reducir a un conjunto parametrizado.
\begin{itemize}
\item Consideramos $\mathcal{P} = \{ \mathbb{P}_{\theta} \colon \theta \in \Theta\}$
\item $\theta$ se llama parámetro del modelo
\item $\Theta$ es el espacio de parámetros
\end{itemize}

Una misma familia $\mathcal{P}$ puede admitir varias parametrizaciones(veremos ejemplos). Idealmente preferimos parámetros interpretables.\\
Otro asunto es la identificabilidad, que consiste en pedir  que $\theta \mapsto \mathbb{P}_{\theta}$ es biunívoca.

\begin{ej} Veamos el estudio de una moneda, en términos de inferencia.\\
Sea $\theta \in \left[0,1\right]$ es la probabilidad de que una moneda (fija) salga cara; suponemos $\theta$ conocido y codificamos cara 1 y sello 0.\\
Se lanza la moneda $n$ veces independientemente, por lo tanto observamos un $x = (x_{1}, x_{2}, \ldots, x_{n}) \in \{0,1\}^n = \mathfrak{X}$.\\

Sabemos que la probabilidad de cara es $\theta$ y que los lanzamientos son independientes $\forall \theta \in \left[0,1\right] = \Theta$.\\
Postulamos que la ley del vector $X = (X_{1}, X_{2}, \ldots, X_{n})$ que genera $x$ está dada por $\mathbb{P}_{\theta}$ tal que :
\begin{align*}
\mathbb{P}_{\theta}\left[X_{i} = 1\right] = \theta, &\quad \forall \theta \in \Theta, i = 1,2,\ldots,n\\
\mathbb{P}_{\theta}\left[X_{1}=x_{1}, X_{2}=x_{2}, \ldots, X_{n}=x_{n}\right] = \mathbb{P}_{\theta}\left[X_{1}=x_{1}\right]\mathbb{P}_{\theta}\left[X_{2}=x_{2}\right]\ldots \mathbb{P}_{\theta}\left[X_{n}=x_{n}\right], &\quad \forall \theta \in \Theta, \forall x_{1}, x_{2}, \ldots, x_{n} \in \{0,1\}
\end{align*}

Recordamos una fórmula muy útil:
\begin{align*}
&\mathbb{P}_{\theta}\left[X_{i} = x_{i}\right] = \theta^{x_{i}}(1-\theta)^{1-x_{i}}, &\quad x_{i} \in \{0,1\}\\
\Rightarrow & \mathbb{P}_{\theta}\left[(X_{1}, X_{2}, \ldots, X_{n}) = (x_{1}, x_{2}, \ldots, x_{n})\right] = \theta^{\sum{x_{i}}}(1-\theta)^{n-\sum{x_{i}}} &\quad \theta \in \Theta = \left[0,1\right],\ x \in \{0,1\}^n
\end{align*}
\end{ej}
Algunos problemas estadísticos que estaremos trabajando son:
\begin{itemize}
\item Estimar $\theta$ o una función de $\theta$ (problema de estimación puntual o por conjunto)
\item Decidir si la moneda es equilibrada (problema de test de hipótesis)
\end{itemize}
\end{document}